\documentclass[a4paper, 11pt]{article}

\usepackage{amsmath}

\title{A Planar Continuous Rock-Paper-Scissors Game}
\author{Ivan Oreshnikov \\ oreshnikov.ivan@gmail.com}

\begin{document}

\maketitle

\section{Introduction}

There exists a simulation somewhere on the web, where a swarm of rocks, a swarm of papers, and a swarm of scissors move around the plain in a Brownian motion and collide with each other. And every collision works as follows
\begin{itemize}
    \item{in each collision there is a winner and a loser particle, who are determined by the rules of the standard ``rock-paper-scissors'' game}
    \item{the loser particle is converted into a winning particle and the simulation continues.}
\end{itemize}
In other words, there exists three types of interactions:
\begin{align*}
    \text{rock} + \text{paper} &= 2 \cdot \text{paper} \\
    \text{paper} + \text{scissors} &= 2 \cdot \text{scissors} \\
    \text{scissors} + \text{rock} &= 2 \cdot \text{rock}
\end{align*}
The simulation is mildly amusing to watch as is. But one can also wonder, how would the same process work if we were to replace the discrete particles with a smooth continous approximation. How would the same process work, if instead of three swarms we had three ``liquids`` on a plane?

\section{Mathematical model}

To work with the continous approximation we need to represent the distribution of particles on a plane with a smooth density function. Since we have three kinds of particles, we need to introduce three different density functions, namely:
\begin{itemize}
    \item{$r(x, y; t)$ for rocks}
    \item{$p(x, y; t)$ for paper}
    \item{$s(x, y; t)$ for scissors.}
\end{itemize}
Now since every particle in a liquid is a constant state of Brownian motion, we can say that in a collision-free environment each density function should behave according to the diffusion equation
\begin{equation*}
    \partial_{t} f + \frac{1}{2} \Delta f(x, y; t) = 0,
\end{equation*}
where $\Delta$ is the Laplacian $\partial^2_{xx} + \partial^{2}_{yy}$. Taking the collisions into account we can write the following system of equations
\begin{subequations}
    \label{eq:PDESystem}
    \begin{align}
        \partial_{t} r + \frac{1}{2} \Delta r = - p \, r + s \, r \\
        \partial_{t} p + \frac{1}{2} \Delta p = - s \, p + r \, p \\
        \partial_{t} s + \frac{1}{2} \Delta s = - r \, s + p \, s.
    \end{align}
\end{subequations}

\section{Numerical methods}

\eqref{eq:PDESystem} is a system of nonlinear partial differential equations. They are notoriously hard to reason about. Nevertheless we will try in the sections that follow. But first let us try and propose a numerical approach the equations.

\end{document}
